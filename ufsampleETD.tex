%&latex

% Define Document Class to be used and options %
\documentclass[12pt,dvipdfm,final,CPage]{ufthesis}
\renewcommand{\rmdefault}{ma1}
%-------------------------------------C:\Program Files\MiKTeX 2.5\miktex----------------------------------%
% Preamble %

% Define Packages To be used and options %
% here you define all the packages you wish to use in your paper, the ones shown are not all necessary,
% but all have purpose and can be very useful, so leave these as default and add packages as necassary
\usepackage[dvipdfm]{graphicx}
\usepackage{amsmath}
\usepackage{amsthm}
\usepackage{url}
\usepackage[letterpaper,hmargin=1in,vmargin=1in]{geometry}
\usepackage{lscape}
\usepackage{hanging}
\usepackage{longtable}
\usepackage{amsfonts}
\usepackage{amssymb}
\usepackage[cmbright]{sfmath}
\usepackage{subfigure}
\usepackage{rotating}
\usepackage{calc}
\usepackage{setspace}
\usepackage{sfmath}%delete or comment this package if using Times New Roman
\usepackage{ufenumerate}
\usepackage{latexsym}
\usepackage{epsf}
\usepackage{epsfig}
\usepackage{euscript}
\usepackage[format=hang,justification=raggedright,singlelinecheck=0,labelsep=period]{caption}
\usepackage[numbers,sort&compress]{natbib}
%\usepackage[authoryear]{natbib}
\usepackage{hypernat}
\usepackage[dvipdfm,hyperfootnotes=false]{hyperref}
%\usepackage[dvips,hyperfootnotes=false]{hyperref}
\hypersetup{colorlinks=true,linkcolor=blue,anchorcolor=blue,citecolor=blue,filecolor=blue,urlcolor=blue,bookmarksnumbered=true,pdfview=FitB} %


%\allowdisplaybreaks

% Prevent figures, tables or algorithms from using a separate page or column alone
\renewcommand{\topfraction}{0.85}
\renewcommand{\textfraction}{0.1}
\renewcommand{\floatpagefraction}{0.75}

% *** Do not adjust lengths that control margins, column widths, etc. ***
% *** Do not use packages that alter fonts (such as pslatex).         ***
% There should be no need to do such things with IEEEtran.cls V1.6 and later.
% correct bad hyphenation here
%\hyphenation{op-tical net-works semi-C:\Program Files\MiKTeX 2.5\miktexconduc-tor}

%------------------------------------------%

% Extra commands or misc formatting such as page alignment or output paper-size commands

\include{extraparameters}

%------------------------------------------%

% Define student-specific info (self-explanatory) %
% Set your personal and paper information
\SetFullName{David Isaac Wolinsky}%
\SetThesisType{Dissertation}%Proposal}%Tutorial}%{dissertation} %{thesis}
\SetDegreeType{Doctor of Philosophy}% {Master of Science}
\SetGradMonth{August}
\SetGradYear{2011}
\SetDepartment{Electrical and Computer Engineering}%
\SetChair{Renato Figueiredo}%
%\SetCochair{John W. Carver III}%uncomment this line and enter the name of your cochair inside the braces if you have one.
%If you have a cochair there two places in the ufthesis.cls file that will need to be uncommented as well
%In the "getting personal information" section about line 630
%And the "Abstract" Section around line 556
% Type your title here in all CAPS %
\SetTitle{DESIGN, IMPLEMENTATION, AND APPLICATIONS of \newline PEER-TO-PEER VIRTUAL PRIVATE NETWORKS \newline FROM GRIDS TO SOCIAL NETWORKS}


%------------------------------------------%

% user defined commands in order to geC:\Program Files\MiKTeX 2.5\miktexnerate new commands, macros, and redefine default commands %
\include{usersetcommands}

%-------------------------------------------------------------------------------------------------------%

% Begin Main Part of Document %

%\renewcommand{\rmdefault}{cmss}
%\renewcommand{\rmdefault}{ma1}
%\renewcommand{\sfdefault}{ma1}
%\renewcommand{\rmdefault}{mns}
%\renewcommand{\rmdefault}{ma1}
%\renewcommand{\rmdefault}{mns}
%\renewcommand{\rmdefault}{ma1}
%\renewcommand{\rmdefault}{cmss}
%\renewcommand{\rmdefault}{ma1}
%\renewcommand{\sfdefault}{cmss}
%\renewcommand{\rmdefault}{@calibri}
%\renewcommand{\rmdefault}{@hatten}
%\renewcommand{\rmdefault}{@lsans}
\begin{document}

%\bibliographystyle{plain}
%\bibliographystyle{ufinit}
%\bibliographystyle{abbrvnat}
%\bibliographystyle{plainnat}
%\bibliographystyle{unsrtnat}
%\bibliographystyle{Chicago_Web}
%\bibliographystyle{apa-good}
%\bibliographystyle{uf_econ}
%\bibliographystyle{Science_Web}
%\bibliographystyle{unsrturl_uf}
%\bibliographystyle{abbrvurl_uf}
%\bibliographystyle{alphaurl_uf}
%\bibliographystyle{ecology_web}
\bibliographystyle{mla-good}
%\bibliographystyle{mla_web}
%\bibliographystyle{plainurl_uf}
%-----------------------------------------------------------------------%

\maketitle %
\makecopyright

%------------------------------------------%

\dedication{% Add your text for the dedication here between the center tags
\addvspace{4.25in}
\begin{center}
I dedicate this to family and those whose have supported me.\\
\end{center}
}

%------------------------------------------%

% Make sure to keep the text within the brackets and the output should turn out correct
\acknowledge{%
Thanks will be given in due time.
}
 %

%------------------------------------------%

% This file includes the file which creates the table of contents %
\include{TOC} %

%------------------------------------------%

%%This is an optional file. A list of abbreviations is NOT even suggested.
%%Best practice is to define the item the first time it is used in the document
%\include{abbreviations}


%------------------------------------------%
% This line adds the word CHAPTER to the TOC just before the listing of the chapter and subsections begins
\addtocontents{toc}{\protect\addvspace{10pt}\noindent{CHAPTER}\protect\hfill\par}{}% This extra line adds the word CHAPTER to the table of contents %
\phantomsection
% Write in only the text of your abstract, all the extra heading jargon is automatically taken care of
\begin{abstract}
Virtual private networks (VPNs) enable existing network applications to run
unmodified in insecure and constrained environments by creating an isolated and
secure virtual environment providing all-to-all connectivity for VPN members.
VPNs are traditionally centralized, limiting applicability due to resource
requirements and the incurred overhead of having all communication traversing
the central server.  In recent years, there has been a paradigm shift towards
the use of peer-to-peer (P2P) communication in VPNs enabling direct
communication amongst participants, alleviating pressure placed upon the
central server relegating it to handling session management and supporting
indirect communication by relaying traffic when NAT(Network Address Translation)
or firewall traversal fails.  Unstructured P2P, though less studied, removes all
centralization from the VPN with the cost of users having to manually create
links amongst peers.  An issue with existing VPN approaches is the lack of
methods for certificate and certificate revocation handling and distribution
requiring the user to employ their own solutions.

I propose a novel VPN using structured P2P overlays emphasizing their scalable
and resilient nature and supporting a security framework that self-configures
through a group-based public key infrastructure for user-friendly operation.
Structured overlays enable users to deploy and maintain decentralized,
distributed applications with limited need for centralization by ensuring fast
seek time (typically $O(\log(N))$) when searching for a node near a location
in the overlay.  The novel contributions of my work include a virtual network
endpoint architecture supporting different aspects of reliability, performance,
portability, and usability; a P2P security architecture integrating PKI (Public
Key Infrastructure) with systems constrained by NATs and firewalls;
user-assisted automated PKI enabled by group-oriented Web 2.0 interfaces; and
the creation of virtual private overlays from existing public overlays to
provide isolation and overlay security for the VPN.  Furthermore, I plan on
investigating how to support VPNs in environments where users lack
administrative permissions through a sockets proxy using overlay-aware TCP,
connecting those hosts to other VPN clients.

The premise for this work is to enhance the usability of VPN systems for
non-expert users in home, small/medium business, and education
environments as demonstrated by prototypes found in the SocialVPN, GroupVPN,
Grid Appliance, and Archer projects.  The SocialVPN creates user-centric VPNs
so that peers only have VPN links with their social network friends, whereas
the GroupVPN employs a group infrastructure to manage VPN members and
distribute VPN configuration.  A free GroupVPN bootstrapping environment
relying on PlanetLab hosted resources has been available for over a year with
already over hundreds of users.  GroupVPN has been used by several universities
to connect decentralized grids.  The Grid Appliance uses the GroupVPN to form
ad-hoc and decentralized computing pools, facilitating computer architecture
research in the Archer project.  The Archer project has been accessed by
student at several universities and has accumulated over 150,000 CPU hours in a
little more than a year.  In addition, the Grid Appliance has been used as both
a teaching tool in distributed computing classrooms as well as by external
users to create their own grids.
\end{abstract}
 %

%-----------------------------------------------------------------------%

% This section encompasses the main body of the paper from all the content through to the biographical sketch

% Chapters to be included (more can be added by creating a new chapter#.tex %
% file and then implementing the /inlcude{chapter#.tex} command as seen below %
%\chapter{Introduction}
\label{introduction}
A Virtual Private Network (VPN) provides the illusion of a local area network
(LAN) spanning a wide area network (WAN) infrastructure by creating encrypted
and authenticated, secure\footnote{For the remainder of this proposal, unless
explicitly stated otherwise, security implies encryption and authentication.}
communication links amongst participants.  Common uses of VPNs include secure
access to enterprise network resources from remote/insecure locations,
connecting distributed resources from multiple sites, and establishing virtual
LANs for multiplayer video games over the Internet.  VPNs, in the context of
this proposal, differ from others that provide ```emulation of a private Wide
Area Network (WAN) facility using IP facilities' (including the public Internet
or private IP backbones).  ''~\cite{ip_vpns}.  These style of VPNs connect large
sets of machines through virtual routers to a virtual WAN environment.

As a tool enabling collaborative environments, VPNs can be useful for many
different types of users.  If friends and family require computer assistance
and their computer guru no longer lives nearby, a VPN enables access to the
remote machine despite networking constraints so long as the user has an
Internet connection.  When traveling abroad, a user may wish that their
Internet traffic be kept private from the local network, a VPN can be
used to route all Internet packets through the users home network, ensuring
the user's privacy.  Many computer and video games have multiplayer networking
components that require direct connectivity and even modern games with
centralized gaming components eventually are no longer supported, players of
these games can continue playing with their remote friends through VPNs.  Small
and medium businesses may find VPNs useful for connecting desktops and servers
across distributed sites securing traffic to enterprise networked resources.
independent organizations that each have their few of their own or no resources
can combine together their resources through a VPN to create a powerful
computing grid.

There are various VPN architectures that attempt to deal with the challenges
presented in these use cases.  In some cases one VPN approach may work,
where another is not applicable, and in some no current VPN approach is
applicable.  In general VPNs face the following challenges:  
\begin{itemize}
\item \textbf{Configuration}:  Initial setup of the VPN.  Where will VPN
resources be located, what type of security credentials will be used, what are
the network parameters, how will users connect to the VPN.
\item \textbf{Management}:  As peers and external resources desire to join the
system, security credentials need to be provided to both.  External resources
need to be linked to the rest of the system.  Occasionally peers misbehave, in
these situations, peers must have their membership revoked.
\item \textbf{Connectivity}:  Peers may want to connect to a remote environment
or to each other.  Communicating through a central resource may create
bottlenecks, but doing so directly may be impossible due to restrictive network
environments.
\item \textbf{Privacy}:  When using a VPN, peers assume that their communication
is private.  VPNs that establish their links through a centralized system are
susceptible to man-in-the-middle attacks, though setting up decentralized
systems can be significantly more complicated.
\item \textbf{Permissions}:  Users must be administrators or given the ability
to run a VPN by an administrator.  Strict environments such as computing labs
or in environments with existing VPNs may prevent the user from being able
to use their own VPN.
\end{itemize}

The key to using a VPN in collaborative environments is making it user-friendly
and scalable.  Applying these requisites to the challenges:  a collaborative
VPN should be easy to configure, users need not be experts in operating systems
(OSs) or networks; a VPN should not rely on any one site or institution to
provide connectivity for the entire VPN; adding new users and resources should
be straight-forward using approaches familiar to common users; peers should be
able to connect to each other directly if and when possible; not only should
the communication in the system be secure but the system providing the VPN
should be secure; and users should be able to connect to the VPN so long as
there is Internet connectivity.  While existing VPN are able to meet some of
these requirements, they are unable to meet them all.  Centralized approaches
(e.g.  OpenVPN~\cite{openvpn}) by their very nature require dedicated
infrastructures and do not allow direct communication between peers though are
the only VPN approach to full tunnel operation and guarantee all-to-all
communication regardless NAT and firewall conditions.  P2P-based approaches
(e.g. Hamachi~\cite{hamachi}, Wippien~\cite{wippien}, Gbridge~\cite{gbridge},
PVC~\cite{pvc}) are vulnerable to man-in-the-middle attacks if session
management is handled by an external provider, rely on a central resource for
the creation of VPN links, and require centralized relays if direct peer
communication across NATs and firewalls fails.  Decentralized approaches
require manual configuration of links between members of the virtual network
(e.g., ViNe~\cite{vine}, Violin~\cite{violin}, VNET~\cite{vnet},
tinc~\cite{tinc}).  Existing P2P approaches lack scalability (N2N~\cite{n2n}
and P2PVPN~\cite{p2pvpn}) or are difficult to configure and lack privacy
(I3~\cite{i3}).

The focus of this proposal is in VPNs useful for collaborative environments
through a novel peer-to-peer (P2P) VPN systems.  In this proposal, I will
review the key components of a VPN and either show how existing P2P systems can
be used to support the components or design and implement new features and
systems as necessary.  P2P systems align well with collaborative environments
in large part due to their decentralized and distributed nature, the challenge
in using P2P is ensuring security and scalability.  P2P systems can be used
to implement scalable autonomic and decentralized systems, though when used
in public environments they do not provide very secure environments as they
are easily compromised by malicious users, but the cost of hosting a private
overlay can out weigh the advantages in collaborative environments.  I extend
my work from approaches that use P2P to implement scalable virtual networks,
IPOP~\cite{ipop} and I3~\cite{i3}, finishing the work of my predecessors by
designing and implementing a system that provides privacy or user-friendly
configurability.  At the heart of my contribution are methods enabling secure,
user-friendly VPNs through the use of P2P systems.

\section{Virtual Private Network Basics}
VPNs consist of two components: clients that communicate with each other and
servers or overlays that provide the infrastructure for clients to find and
establish communication with each other.  From a users perspective the
environment provided by a VPN client is the same regardless of how the server
or overlay is implemented.  The clients interface with the server can also
be abstracted such that clients are quite generic.

Figure~\ref{fig:vpn} abstracts the common features of all VPNs clients, a
service that communicates with the VPN system and a virtual network (VN) device
for host integration.  During initialization, the VPN services authenticates
with the system~\footnote{A system in this context refers any portion of the
VPN system including a central server, another VPN client, or a relay.},
optionally, querying for information about the network, such as network address
space, address allocations, and domain name service (DNS) servers.  At which
point, the VPN enables secure communication amongst participants.

Clients can authenticate with the overlay using a variety of methods.  A system
can be setup quickly by using no authentication or a shared secret such as a key
or a password.  Using accounts and passwords with or without a shared secret
provides individualized authentication, allowing an administrator to block all
users if the shared secret is compromised or individual users who act
maliciously.  In the most secure approaches, each client has a unique signed
certificate making brute force attacks very difficult.  The trade-offs in the
approaches come in terms of security, usability, and management.  While the use
of signed certificates provides better security than shared secrets,
certificates require more configuration and maintenance.  In a system comprising
of non-experts, the usual setup uses a shared secret and individual user
accounts.  Secrets can be packaged with the VPN application, so long as it is
distributed through secure channels such as authenticated HTTPS.

A VN device allows applications to communicate transparently over the VPN.  The
VN device provides mechanisms for injecting incoming packets into and retrieving
outgoing packets from the networking stack, enabling the use of common network
APIs such as Berkeley Sockets, thereby allowing existing application to work over
the VPN without modification.  While there are many different types of VN
devices, TAP~\cite{tap} stands out from the rest due to its open source and
pervasive nature.  TAP allows the creation of one or more Virtual Ethernet and
/ or IP devices and is available for almost all modern operating systems
including Windows, Linux, Mac OS/X, BSD, and Solaris.  A TAP device presents
itself as a character device providing read and write operations.  Incoming
packets from the VPN are written to the TAP device and the networking stack in
the OS delivers the packet to the appropriate socket.  Outgoing packets from
local sockets are read from the TAP device.

VN devices can be configured manually though command-line tools or OS' APIs or
dynamically by the universally supported dynamic host configuration process
(DHCP)~\cite{dhcp0, dhcp1}.  Upon the VN device obtaining an IP address, the
system adds a new rule to the routing table that directs all packets sent to
the VPN address space to be directed to the VN device.  Packets read from the
TAP device are encrypted and sent to the overlay via the VPN client.  The
overlay delivers the packet to another client or a server with a VN stack
enabled.  Received packets are decrypted, verified for authenticity, and then
written to the VN device.  In most cases, the IP layer header remains unchanged,
while VPN configuration determines how the Ethernet header is handled.

\section{Computer Network Architectures}
All models for computer communication in distributed systems fall under two
categories:  centralized and decentralized, though they can be further
divided to allow for self-configuring dynamic systems through the use of P2P
communication.  The architectures commonly used for implementing VPN systems
are:

\begin{itemize}
\item \textbf{Centralized Organization and Communication} - These are client
server systems, where all distributed peers both locally and remote are clients
connecting into a dedicated server resources.  Clients never communicate with
each other directly only, but rather every message between two clients must
traverse the server.  Most online social networks (OSNs) are representative of
these type of systems, users of OSNs like Facebook~\cite{facebook} and
MySpace~\cite{myspace} communicate through centralized environments never
directly to each other's computer.  The issue with these systems is the reliance
on dedicated resources requiring that the server be online for clients to
organize and communicate with each other, thus if a server goes offline or
becomes overwhelmed by clients the system is rendered useless.
\item \textbf{Centralized Organization and Decentralized Communication} - The
first set of P2P popular P2P systems like the original Napster
and Kazaa used this sort of system.  Like the client-server model,
clients connect to a server to find other clients, though instead of
communicating through the server, the clients form direct connections with each
other.  These approaches are limited by network address translation (NAT) and
firewalls that may prevent peers from communicating with each other.  In these
cases though, the central server may act as a relay allowing the two clients to
communicate through it.  Unlike systems using centralized communication, these
systems are less susceptible to being overwhelmed by client traffic and even if
the server goes offline existing client links remain active though new
connections cannot be formed.
\item \textbf{Decentralized with Manual Organization} - To address the issues
of a central system going offline, many distributed servers may be used and
clients can be configured to connect to any number of them creating an overlay.
In these systems, servers are explicitly configured to communicate with other
remote and local servers.  Though this approach improves upon the issues
inherent with completely centralized architectures, if a site goes offline any
systems communicating through it will no longer be connected to the rest of the
system until the administrator creates additional links or the site becomes
active again.  Clients in these systems do not typically form direct links with
each other, rather they route packets through the overlay.  This approach has
been used to create scalable VPNs, like ViNe~\cite{vine}, VNET~\cite{vnet},
Violin~\cite{violin}, and Layer 2 Tunneling Protocol based VPNs~\cite{l2tp}.
\item \textbf{Decentralized with Automatic Organization} - The last model falls
under systems that self-organize.  In this environment, there is no distinction
amongst peers as they act as both client and servers, i.e., a P2P system or
overlay.  P2P systems are usually distributed with list of common peers.  When
attempting to connect with the P2P overlay, a peer randomly selects peers on
this list until it is able to connect with one.  This connection is then used
to form connections with other peers currently in the overlay.  The overlay
can be organized in two different forms: randomly or deterministically
creating unstructured or structured overlays, respectively.  In an unstructured
overlay, links are formed arbitrarily, thus a peer searches for another peer
by broadcasting the message or using stochastic techniques.  Structured
overlays are organized into deterministic shapes, each peer is expected to
have connections to certain other peers forming shapes such as rings and
hypercubes.  Peers can be found deterministically using greedy routing
approaches in usually $O(\log(N))$ time.  Gnutella~\cite{gnutella} file sharing
system and Skype~\cite{skype} are popular examples of unstructured systems,
while P2PSIP~\cite{p2psip} and distributed hash tables (DHTs)~\cite{chord} are
popular in structures systems.  The challenges to unstructured systems is
finding data objects in reasonable amount of time, while structured systems
suffer when large amount of peers join or leave the system, churn.  In general,
both approaches are difficult to secure due to their typical application.
When used in closed environments though, they have been shown to be very
useful, exemplified by Dynamo~\cite{dynamo} or BigTable~\cite{bigtable}.
\end{itemize}
As this proposal will use structured overlays as the foundation in building
scalable, decentralized VPNs, Chapter~\ref{structured_p2p} provides more
detailed review of structured overlays and challenges in decentralized
communication with emphasis on security, establishing connections, and
reliability.  I also provide solution to these challenges in the form of
private virtual overlays bootstrapping secure overlays using public
free-to-join overlays, a decentralized relay system when direct NAT or firewall
traversal fails, and an overlay-aware TCP enabling reliable and efficient
communication over unreliable links.

\section{Contributions}
In the introduction, I presented a list of challenges a VPNs face.  When
applied to collaborative environments, the resulting requirements are 
self-configuring environments enabling even non-experts to setup, deploy,
and manage their own VPNs; peers should communicate with each other
directly when possible though still have reasonably efficient alternatives;
the system should be reliable and ensure the privacy of its users; and
users should be able to access the VPN regardless of their access rights.
To address these requirements, I propose a novel GroupVPN using structured
overlays consisting of the following novel contributions:

\begin{itemize}
\item \textbf{Decentralized Relays}:  In collaborative environments, most peers
will be behind NATs and potentially firewalls as well.  While in general 90\% of
NATs are traversable through existing approaches, not all are and firewalls
complicate the matter.  While these peers can communicate through the overlay,
as the overlay grows, this latency can grow taking seconds for peers to send
a message to each other.  To improve this situation, I propose the creation of
autonomic Te-hop relays between the peers.
\item \textbf{Private Virtual Overlays}:  At any given time, peers may or may
not be connected to the overlay.  Private overlays can consist of varying sizes
of users, in environment where there are very few users, it is possible that
not a single user can provide a dedicated, publicly addressable  resource.  In
this case, the overlay can experience downtime, even though there may be users
behind NATs and firewalls wanting to use the overlay.  To address this issue,
I propose the use of a public free-to-join overlay to bootstrap into a private
overlay.  Peers use the public overlay to find each other and exchange connection
information using secure messages.  Only peers with appropriate security
credentials are able to join the private overlay.
\item \textbf{Overlay Communication Models}: In my experience, when using the
overlay based connections, performance suffers due to being processed by the
overlay's state machine.  I will work towards addressing this issue by
investigating different models for using overlays to establish direct
communication:  communicating through the overlays state machine, bypassing the
overlays state machine but reusing its connection management, and creating
links unused by the overlay.
\item \textbf{Overlay-Aware TCP}: Overlays consist of peers connecting and
disconnecting at random and in order to provide light-weight approaches that
provide reasonable NAT and firewall traversal are connected using UDP.  As such
large streams of data cannot be sent reliably through the overlay.  This is not
an issue when a VPN has administrator permission enabling the reuse of the OS'
network stack including TCP.  For situations that lack this ability, I will
design a TCP stack with focus on efficient and reliable streaming using
overlays.
\item \textbf{Self-Configuring VPN Architectures}: Many existing VPN approaches
require the users to setup their environment and do not provide a plug and play
system.  In addition, different environments call for different types of VPNs,
explicitly, users need their own VPN instances and clusters may benefit from
a single VPN instance for the cluster or may desire fault tolerance of having
many but do not want the communication overhead when talking to VPN peers on
the LAN.  I address this issue with a self-configuring VPN approach that can be
applied to various environments scaling from a single computer to many.
\item \textbf{Userspace Virtual Networking Stack}: To address the case where
users do not have permissions to run VPNs, I plan on designing and implementing
a VPN requires no user permissions and can connect to other VPNs that do.
\item \textbf{P2P Enabled Full Tunnel VPN Mode}: When in insecure environments
such as browsing private information in a coffee shop, users may desire to
prevent local users and administrators from sniffing their traffic.  Traditional
VPNs support this behavior, but the approach is difficult to implement in P2P
systems due to their dynamic nature.  All existing decentralized VPN approaches
lack the ability to perform this behavior.  I propose a method that not only
works for decentralized and P2P systems and ensures a greater level of security
than that provided by existing approaches in centralized systems.
\end{itemize}

To supplement this work, I plan on the investigating the application of these
systems in the realms of online social networking and grid computing.  I will
determine feasibility of implementing an online social network using the 
structured overlays with particular focus on the use of private virtual
overlays as social network profiles or profile overlays.  In addition, the
primary motivation for my work has been the use of self-configuring VPNs and
systems for grid computing.  In this proposal, my culminating work, I will
describe how this system can be used to create a novel self-configuring grid
system that allows users who have limited knowledge of operating systems,
networks, and computers to create their own grids in a matter minutes.

The rest of this proposal is organized as follows.  In
Chapter~\ref{structured_p2p}, I present an in depth analysis of structured
overlays including a review of NAT constraints and traversal and then explore
the use of decentralized relays.  This leads into Chapter~\ref{vpo}, which
discusses security issues in structured overlays and a solution enabling users
to create their own private pools using public overlays.  Chapter~\ref{vpns}
continues the discussion of VPNs with focus on the different architectures 
focusing on both local and network configuration.  Chapter~\ref{gridappliance}
details the construction of grid systems using self-configuring approaches
enabled by virtualization.  In Chapter~\ref{spo}, I describe a method for
creating online social profiles using virtual private overlays.  Finally, I
conclude the paper discussing real systems and future work in
Chapter~\ref{conclusion}.

\section{Figures and Tables}
\begin{figure}[ht]
\centering
\epsfig{file=figs/vpn.png.eps, width=4in}
\caption[A typical VPN client]{A typical VPN client.  A VN device makes
application interaction with the VPN transparent.  Packets going to the VPN
destination are sent by routing rules to the VN device interfaced by the VPN
client.  The VPN client sends and receives packets from other VPN participants
via the hosts physical network device.}
\label{fig:vpn}
\end{figure}


%-----------------------------------------------------------------------%

% Includes appendices called into the appendix file(check the comments regarding editing your appendix in that file)
\include{appendix}

%------------------------------------------%

% Make List of References (BibTeX implemented using the Natbib package)
% un-comment your preferred bibliography style and replace the
% bibliography file "sample" with the name of your .bib file
% REMEMBER!!! If you want un-numbered references comment the Natbib package with
% The numbered options in the packages.tex file and un-comment the package with the authoryear option
% See the included pdfs of the various styles to see the differences.
% The citation style differences are from the \citet{key} and \citep{key} commands
% More options are available; see the Natbib documentation for details


\bibliography{ufsampleETD}
% You can have more than one library of references
%------------------------------------------%

% Bio Sketch %
% Just type your bio in between the brackets
\biography{%
David Isaac Wolinsky was born on October 31, 1982.  He has been married to Donna
Korin Wolinsky since June 2007 and was blessed with a son, Isaac Emmanuel, on
November 30, 2009.  He began his studies at the
University of Florida in August 2001, obtained a bachelors of science degree
in the spring of 2005 and a masters of science degree in the spring of 2007, and
am currently pursuing a doctorate of philosophy.  My advisor is Professor Renato
Figueiredo, whom I have had the pleasure of working with since the spring of 2006
at the Advanced Computing and Informatoin Systems Lab.

My primary research focuses are network virtualization using structured P2P
overlays.  This research work has been realized in IPOP, a free (GPL) network
virtualization software.  I have worked on enabling DHTs, decentralized NAT
traversal through relays, software models for improved network virtualization,
and autonomic virtual networking stacks.  This work is a major contribution to
my other research focus, Grid Appliance, which enables the creation of
decentralized, distributed grids through virtualized, physical, and cloud
resources.

During my free time, I enjoy time with my boy (son), running, playing
basketball with the LBBA (Larsen-Benton Basketball Association) colleagues, and
occassionally playing video games.  At one point prior to engaging in the arts
of seeking a Ph.D., I was ranked in the top 20 on the US East Warcraft III Free
For All Ladder.  Most of my time is split between the Archer project and
attempting to finish my Ph.D. prior to turning 30, thankfully I have a few years
prior to turning 30.  During my deeper moments, I contemplate my struggle with
the amazing gift that Christ as my savior gave to me in spite of my brokeness.
}


%------------------------------------------%

\end{document}

%-------------------------------------------------------------------------------------------------------%
