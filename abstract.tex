% Write in only the text of your abstract, all the extra heading jargon is automatically taken care of
\begin{abstract}
Virtual private networks (VPNs) enable existing network applications to run
unmodified in insecure and constrained environments by creating an isolated and
secure virtual environment providing all-to-all connectivity for VPN members.
VPNs are traditionally centralized, limiting applicability due to resource
requirements and the incurred overhead of having all communication traversing
the central server.  In recent years, there has been a paradigm shift towards
the use of peer-to-peer (P2P) communication in VPNs enabling direct
communication amongst participants, alleviating pressure placed upon the
central server relegating it to handling session management and supporting
indirect communication by relaying traffic when NAT or firewall traversal fails.
Unstructured P2P, though less studied, removes all centralization from the VPN
with the cost of users having to manually create links amongst peers.  An issue
with existing VPN approaches is the lack of methods for certificate and
certificate revocation handling and distribution requiring the user to employ
their own solutions.

I propose a novel VPN using structured P2P overlays emphasizing their scalable
and resilient nature and supporting a security framework that self-configures
through a group-based public key infrastructure for user-friendly operation.
Structured overlays enable users to deploy and maintain decentralized,
distributed applications with limited need for centralization by ensuring fast
seek time (typically $O(\log(N))$) when searching for a node near a location
in the overlay.  The novel contributions of my work include a configurable
virtual network architecture to support different aspects of reliability,
performance, portability, and usability; a P2P security architecture
integrating PKI with systems constrained by NATs and firewalls; user-assisted
automated PKI enabled by group-oriented Web 2.0 interfaces; and the creation of
virtual private overlays from existing public overlays to provide isolation and
overlay security for the VPN.  Furthermore, I plan on investigating how to
support VPNs in environments where users lack administrative permissions
through a sockets proxy using overlay-aware TCP, connecting those hosts to
other VPN clients.

The premise for this work is to enhance the usability of VPN systems for
non-expert users to use in home, small/medium business, and education
environments as demonstrated by prototypes found in the SocialVPN, GroupVPN,
Grid Appliance, and Archer projects.  The SocialVPN creates user-centric VPNs
so that peers only have VPN links with their social network friends, whereas
the GroupVPN employs a group infrastructure to manage VPN members and
distribute VPN configuration.  A free GroupVPN bootstrapping environment
relying on PlanetLab hosted resources has been available for over a year with
already over hundreds of users.  GroupVPN has been used by several universities
to connect decentralized grids.  The Grid Appliance uses the GroupVPN to form
ad-hoc and decentralized computing pools, facilitating computer architecture
research in the Archer project.  The Archer project has been accessed by
student at several universities and has accumulated over 150,000 CPU hours in a
little more than a year.  In addition, the Grid Appliance has been used as both
a teaching tool in distributed computing classrooms as well as by external
users to create their own grids.
\end{abstract}
