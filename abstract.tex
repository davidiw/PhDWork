% Write in only the text of your abstract, all the extra heading jargon is automatically taken care of
\begin{abstract}

Virtual private networks (VPNs) enable existing network applications to run
unmodified in insecure and constrained environments by creating an isolated and
secure virtual environment providing all-to-all connectivity for VPN members.
While there exist both centralized and distributed VPN implementations, current
approaches lack self-configuration and organization capabilities that would
reduce management overheads and minimize effort by non-experts.  Recent use of
peer-to-peer (P2P) techniques have focused on alleviating pressure placed upon
infrastructure nodes by allowing peers to form direct connections for
communication purposes, while infrastructure nodes are used for handling
session management and supporting indirect communication by relaying traffic
when NAT (Network Address Translation) or firewall traversal fails.  In terms
of decentraalized, P2P-based VPN solutions, the mechanisms explored thus far in
related works employ unstructured P2P systems, which can have significant
scalability limitations.  This thesis constructs a novel decentralized P2P VPN
that addresses the following core aspects that are integral to
user-friendliness: bootstrapping, discovery, security, and endpoint
configuration.

A resource joining a distributed system goes through a bootstrapping process.
The target environment for VPNs include small systems with many if not all
users behind NATs and firewalls making the bootstrapping process challenging.
Centralized systems address the bootstrapping problem by using a common
resource for peer registration, discovery, and connection establishment.
Centralized systems, however, come with additional costs in deploying and
managing a dedicated resource with a public Internet address and the capability
to handle demands placed upon it by clients.  I have investigated, implemented,
and evaluated decentralized means to bootstrap private P2P overlays for
connectivity-constrained resources, with an approach that supports a recursive
overlay organization or the use of third-party free-to-join public overlay
infrastructures using technologies such as XMPP.

Bootstrapping helps establish connectivity into an overlay; however, many
systems including P2P VPNs require a means for discovery specific peers.
Existing VPNs either rely on large tables hosted on infrastructure nodes or
overlay broadcast techniques to find a resource. As a system grows in capacity,
these approaches have their limitations, especially in VPNs where all IP
addresses are independent of their location inside the VPN.  I have employed
distributed hash tables to efficiently establish decentralized IP address
allocation and discovery seamlessly providing scalability and resilience.

In a VPN, other peers are typically either trusted directly by the peer, or
indirectly through a trusted third-party.  While users may trust a third-party
to assist them in creating network links to other peers, they do not desire to
have intermediaries that are able to read or modify their IP packets.
Unfortunately, most VPNs only encrypt messages on a point-to-point (PtP) basis
allowing these intermediaries privileged access to their identity and their
messages.  In these cases, end-to-end (EtE) security relies on out-of-bound
exchanges and applications.  To transparently handle security at both PtP and
EtE layers across a wide spectrum of communication transports, I have developed
a novel security filter, which has been demonstrated to support existing Public
Key Infrastructure based security systems (such as DTLS) for both PtP and EtE
traffic inside connectivity-constrained environments.

While security primitives enable private and authenticated communication, the
configuration and management overheads involved in establishing trust and
maintaining secure connections in VPNs are a significant hindrance to usability
and adoption.  In my approach, all security links are established from
exchanged certificates, so each peer is uniquely identifiable.  My approach
uniquely handles administrative and user aspects of certificates automatically
through the use of online social networking features such as peer relationships
and groups.

The above self-organizing mechanisms to create VPN links need to be
complemented with approaches that support effective bindings to endpoints from
which messages are captured/injected from/to the VPN.  In a typical approach,
called the interface model, each resource in the VPN has a local binding to the
VPN by locally installed software.  Unfortunately, this introduces significant
overheads when two or more such systems are running inside the same trusted
LAN.  Alternatively, if all resources in a LAN connect to a common VPN, such as
in a grid or for cloud computing environments, the resources can share a common
entry point to the VPN through a router model.  Unfortunately, existing
approaches do not transparently configure the router and connected resources.
Additionally, the router model does not work well on shared networks, where
there are either untrusted users or some resources should not be accessible
through the VPN.  I have shown herein how all of these considerations can be
handled without the introduction of new protocols by utilizing existing
services commonly provided by network stacks, primarily DHCP and ARP, which
enables a new type of VPN model that balances the benefits of the interface and
router models.

The premise for this work is to enhance the usability of VPN systems enabling
wider adoption by non-expert users in home, small/medium business, and
education environments.  The concepts for this work have been carefully
designed, implemented, and evaluated and then demonstrated through the
implementation of novel systems (SocialVPN, GroupVPN, and Grid Appliance)
accessed by real users.  The SocialVPN creates user-centric VPNs so that peers
only have VPN links with their social network friends, whereas the GroupVPN
employs a group infrastructure to manage VPN members and distribute VPN
configuration.  A free GroupVPN bootstrapping environment relying on PlanetLab
hosted resources has been available for over three years and has been accessed
by over hundreds of users including several universities and commercial
entities, whereas the SocialVPN has over 80 active members online at any given
time.  The Grid Appliance uses the GroupVPN to form ad-hoc and distributed
computing pools, facilitating computer architecture research in the Archer
project.  The Archer project has been accessed by student at several
universities and has accumulated over 500,000 CPU hours in a little less than
three years.  Furthermore, the Grid Appliance has been used as both a teaching
tool in distributed computing classrooms as well as by external users to create
their own grids.  The challenges faced in these deployments have opened the
door for other avenues of research into built-in self-simulation, P2P
connection establishment, efficient IP broadcasting and multicasting, and
decentralized establishment of Internet gateways.

\end{abstract}
