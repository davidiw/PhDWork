% Write in only the text of your abstract, all the extra heading jargon is automatically taken care of
\begin{abstract}
Structured peer to peer overlays enable users to deploy and maintain
decentralized, distributed applications with limited need for centralization.
Structured overlays are constructed in such a way to guarantee fast seek time
when searching for the node nearest to a specific overlay address. A simple
data store known as a distributed hash table (DHT) can be built on top of this
to provide the storage of key, value pairs. Each key consists of an overlay
address, which can be produced by using a cryptographic hashing function to
generate an overlay address, uniformly distributed values into the overlay.
Applications can use the DHT in place of a centralized directory. In this
proposal, I present applications and enhancements to structured overlays that
can be made decentralized using this approach. The key contributions in the
field of structured overlays, I have and propose to research are enabling
public key infrastructure and decentralized revocation; enabling application
specific parallel, private overlays; handling situations where direct
connection between peers is hindered by NATs and firewalls; and reliable
streaming.

The focal applications for this work are virtual private networks (VPNs),
social networks, and grid computing. For each of these, I explain how to apply
structured overlays in order to limit the use of central and third-party
resources making each application user focused. Through these examples, I argue
that the techniques used in this proposal can be used to enhance structured
overlays applicability and ease of use. 
\end{abstract}
