\chapter{Social Profile Overlays}
\label{spo}
Online social networking has become pervasive in daily life, though as social
networks grow so does the wealth of personal information that they store.  Once
information has been released on a social network, known as a user's profile,
the data and the user are at the mercy of the terms dictated by the social
network infrastructure, which today is typically third-party, centrally owned.
If the social network engages in activities disagreeable to the user, due to
change of terms or opt-out programs not well understood by users such as recent
issues with Facebook's Beacon program~\cite{facebook_beacon}, the options
presented to the user are limited: to leave the social network (surrendering
their identity and features provided by the social network), to accept the
disagreeable activities, or to petition and hope that the social network
changes its behavior. 

As the use of social networking expands to become the primary way in which users
communicate and express their identity amongst their peers, the users become
more dependent on the policies of social network infrastructure owners.  Recent
work~\cite{p2p_socialnetwork} explores the coupling between social networks and
P2P systems as a means to return ownership to the users, noting that a social
network made up of social links is inherently a P2P system with the aside that
they are currently developed on top of centralized systems.  In this paper, we
extend this idea with focus on the topic of topology; that is, how to
self-organize social profiles that leverage the benefits offered by a
structured P2P overlay abstraction.

Structured P2P overlays provide a scalable, resilient, and self-managing
platform for distributed applications.  Structured overlays enable users to
easily create their own decentralized systems for the purpose of data sharing,
interactive activities, and other networking-enabled activities.  This work
extends Private Virtual Overlays described in Chapter~\ref{vpo}.  This chapter
expands upon this approach with in-depth discussion on how to apply this
technique to enable social network overlay profiles.

Social networks consist of users, each has a profile, a set of friends, and
private messaging.  The profile contains user's personal information, status
updates, and public conversations, similar to a message board.  Friends are
individuals trusted sufficiently by a user to view the user's profile.  Private
messaging enables sending messages discretely between users without leaking the
message to other members.  Using this model, a public overlay can be used as a
directory for finding friends and joining existing profile overlays.  Each user
has their own profile overlay, secured via a public key infrastructure (PKI) to
which they are the certificate authority (CA).  The profile overlay stores
profile data in its distributed data store, supporting profile access in
scalable mechanisms regardless of the profile owner's online status.  This
chapter explains architecture of these overlays, presented in
Figure~\ref{fig:system}, how they are used to find and befriend peers, and
describes approaches to handling profile data and establishing initial
connections to profile overlays.

\section{Peer-to-Peer Social Networks}
In~\cite{peerson}, a DHT provides the look-up service for storing meta data
pertaining to a peer's profile. Peers query the DHT for updated content from 
their friends by hashing their unique identifiers (e.g. friends' email
addresses).  The retrieved meta data contains information for obtaining the
profile data such as IP address and file version. Their work relies
on a PKI system that provides identification, encryption, and access control.
In contrast, the described approach provides each user their own private
overlay secured by point-to-point encryption and authentication amongst all
peers in the profile overlay.  The profile overlay provides a clean abstraction
of access control, whereby once admitted to a private overlay, users can access
a distributed data store which holds the contents of the owners profile.

\cite{vis-a-vis} takes a different approach by depending on virtual individual
servers (VIS) hosted on a cloud infrastructure such as Amazon EC2. Friends
contact each other's VIS directly for updates.  A DHT is used as a directory for
groups and interest-based searches. Their approach assumes bidirectional
end-to-end connectivity between each VIS, where a profile is only available
during the up time of the VIS.  Because of the demands on network connectivity
and up time, the approach assumes a cloud-hosted VIS and has difficulty being used on user-owned resources.
Our approach enables users to avoid the need for all-to-all connectivity and
constant up time through the use of NAT traversal support and the
ability to store the profile in the overlay's distributed data store.

The approach presented in~\cite{matryoshka} also uses a DHT for looking up a peer's
 circle of friends.  Once a node in the peer's
outermost circle is found, that node is used to route profile requests to the
innermost circle which contains replicas of a peer's profile. Trust is enabled
through the use of an identification service contacted through the DHT.  The
circle of friends concept lacks the simplicity of the abstraction made in the
described approach, which can easily be applied to existing structured overlays
unlike the concept of innermost and outermost circles.  Our approach
also enables the profile owner to serve as a CA, ensuring that nodes can only
access a profile overlay after having obtained a signed certificate.  

Unlike the above approaches, the P2P social network presented in~\cite{tribler-osn}
uses an unstructured overlay without a DHT where peers connect directly to
each other rather than through the overlay establishing unique identifiers to
deal with dynamic IPs.  Peers cache each other's data to improve availability.
While helper nodes are used to assist with communication between peers behind
NATs.  The approach lacks security and access control considerations and lacks the
guarantees and the simplicity of the abstraction offered by a structured overlay.

\section{Social Overlays}
\label{social_overlays}
This section explains how to map online social networking to a
multi-overlay social network consisting of a public directory overlay with many
private profile overlays.  The directory overlay supports friend discovery and
verification and stores a lists of peers currently active in each profile
overlay.  Profile overlays support message boards, private messages, and media
sharing.

\subsection{Finding and Verifying Friends}
In a traditional social network, a directory provides the ability to search
for users using public information, such as the user's full name, user ID,
e-mail address, group affiliations, and friends.  The search results return zero
or more matching directory entries.  Based upon the results, the user,
\textit{A}, can potentially make a friendship request.  The request receiver,
\textit{B}, can review the public information of A to making a decision.  If
\textit{B} accepts the request, \textit{A} and \textit{B} are given access to
each other's profiles.  Once profile access has been enabled, the users can
learn more information, and if it turns out to be a mistake, the peers can
unilaterally end the relationship.

To map this to the proposed social overlay, the directory entries would be
inserted into the DHT of a public overlay.  As discussed in previous work, the
DHT keys for these entries should consist of a subset of the user's public
information in lower-case format and hashed to an overlay  address.  The value
stored at these keys is the user's certificate, which consists of its public
information and an overlay address where the user expects to receive
notifications.  This overlay address can be used for asynchronous offline
messaging, whose function will be explained shortly.

Because the users need a way to verify each other that involves social
credentials, a new form of certificate can be used.
The main portion of the certificate is similar to a self-signed
x509 certificate with public information such as user's name, e-mail
address, and group affiliations embedded into the certificate.  At the tail of
the certificate is a friend list represented by many friend entries.  This can
be done by employing a technique similar to PGP: users can acquire from their
friends a signed message consisting of a hash of the peer's certificate, the
time stamp, and the friend's certificate hash signed by the friend.  Since PGP
does not provide a strong method for revocation, the time stamp provides
additional information to help decide whether or not a friendship link is still
active without accessing the profile overlay of either peers.  Peers should
request a new friend list entry within a certain period of time or it will
appear that the friendship is no longer valid.

While looking for an individual, a peer may discover that many individuals have
overlapping public information components, such as the user's name.  Assuming
all entries are legitimate, the overlay must have some method of supporting
multiple, distinct values at the same key, leaving the peer or the peer's DHT
client to parse the responses and determining the best match by reviewing the
contents of each certificate.  Alternatively, a technique like
Sword~\cite{sword}, which supports distributing the data across a set of nodes
and using a bounded broadcast to discover peers that match all information,
could be used for searching.

If a peer, \textit{A}, desires a friendship with another peer, \textit{B},
\textit{A} issues a friendship request, which will be stored in the DHT
at the overlay address listed in \textit{B}'s certificate, as described earlier.
The friendship request consists of the self-signed certificate of
\textit{A}, the requesting peer; the public information of the request receiver,
\textit{B}; and a time stamp; all signed with the private key associated with 
\textit{A}'s private key matched to their self-signed certificate.
% Though because DHTs are soft state systems having
% leases, the requester must reinsert the request upon timeout and no response for
% the receiver.

Within a reasonable amount of time after a request has been inserted into the
DHT, \textit{B} can come online and check for outstanding requests.  Upon
receiving a request, \textit{B} has three choices: a conditional accept, an
unconditional accept, or a reject.  During an unconditional accept, \textit{B}
signs \textit{A}'s request and issues a request to befriend \textit{A}.
Alternatively in the case of a conditional accept, \textit{B} issues a friendship
request, waits for a reply, and investigates the profile prior to signing the
\textit{A}'s request.  Once a user has received a signed certificate,
they may access the remote peer's profile overlay as discussed
in~\ref{profile_overlay}, which is also responsible for activities such as
revocation.

%Since a DHT is a soft-state system that uses leases to remove expired data, requests
%and responses must be occassionally reinserted into the DHT.  Alternatively, the
%approaches mentioned in~\cite{} that suggest methods of storing data in overlays
%using quotas could be used to ensure fair usage of the overlay.

\subsection{The Profile Overlay}
\label{profile_overlay}
In a traditional social network, the profile or user-centric portion consists
of private messaging, data sharing, friendship maintenance, and a public
message board for status updates or public messages.  This section explains how
these components can be applied to a structured overlay dedicated to an
individual profile.

Using the techniques such as those described in Chapter~\ref{vpo}, it is feasible
to efficiently multiplex a P2P system across multiple, virtual private overlays enabling
each profile owner to have a profile overlay consisting of their online friends.
For access control, a PKI is employed, where each member uses the signed certificate
generated during the ``finding and verifying friends'' stage.  All links are
encrypted using symmetric security algorithms established through the PKI,
thus preventing uninvited guests from gaining direct access to the overlay and
hence the profile.  Because the profile owner also is the CA for all members of
the overlay, they can easily revoke users from access to the profile overlay.
Chapter~\ref{vpo} describes mechanisms for overlay revocation through the
use of broadcasting for immediate revocation and the use of DHT for indirect
and permanent revocation.

The message board of a profile can be stored in two ways: distributed within the
profile overlay via a data store or stored on the profile owner's personal
computing devices.  The distributed data store provide the profile when the
owner is offline and also distributes the load for popular profiles.  For
higher availability, each peer should always be a provider for all data in their
profile when they are online.  To ensure authenticity and integrity, all peers
should sign their messages and each peer's certificate should be available in
the overlay for verification.  Messages that are unsigned should be ignored
by all members of the overlay.  An ideal overlay for this purpose should
support complex queries~\cite{complex_queries} allowing easy access to data
stored chronologically, by content, by type, i.e., media, status updates,
or message board discussions.

Private messaging in the profile overlay is unidirectional meaning that only
the profile owner can receive private messages using their overlay.  To
enforce this, a private message should be prepended with a symmetric key
encrypted by the profile owners public key, the message should be appended
by a hash of the message to ensure integrity and the entire message encrypted
by the symmetric key.  This approach ensures that only the sender and the
profile owner can decrypt the private message.  The contents of the private
message should include the sender, time sent, and the subject.  Messages can
be stored in well known locations, so that the profile owner can either poll
the location or, alternatively, use an event based system to notify them of
the new message.

\subsection{Active Peers}
The directory overlay should be used to assist in finding currently active peers
in the profile overlays.  By placing their node IDs at a well-known, unique
per-profile overlay keys in the DHT, active peers can bootstrap incoming peers
into the profile overlay.  Because the profile overlay members all use PKI to
ensure membership, even if malicious peers insert their ID into the active
list, it would be useless as the peer would only form connections with peers
who also have a signed certificate.

\section{Challenges}
\label{outstanding}
While structured P2P overlays have been well-studied in a variety of applications,
their use in social profile overlays raises new interesting questions, including:

{\bf 1) Handling small overlay networks} - P2P overlay research typically focuses on
networks larger than the typical user's friend count (Facebook's average is
130\footnote{http://www.facebook.com/press/info.php?statistics}).  Because social profile overlays are comparatively smaller, this can
impact the reliability of the overlay and availability of profile data.  A user
can host their own profile; however when the user is disconnected it is important
that their profile remains available even under churn. It is thus important to
characterize churn in this application to understand how to best approach this
problem. An optional of per-user deployment of a virtual individual server (VIS)
and the use of replication schemes aware of a user's resources provide possible
directions to address this issue.

{\bf 2) Overlay support for low bandwidth, unconnected devices} - devices such as
smart phones cannot constantly be actively connected to the overlay and the
connection time necessary to retrieve something like a phone number may be
too much to make this approach useful.  Similar to the previous challenge,
this approach could benefit from using a VIS enabling users access to their
social overlays by proxy without establishing a direct connection to the overlay
network.

{\bf 3) Reliability of the directory and profile overlay} - Overlays are
susceptible to attacks that can nullify their usefulness.  While
the profile overlay does have point-to-point security, in the public,
directory overlay, the lack of any form centralization makes policing the system
a complicated procedure.  While the approach of appending friends list can assist
users in making decisions on identity, it does not protect against denial of
service attacks.  For example, users could attempt create many similar identities
in an attempt to overwhelm a user in their attempt to find a specific peer.
Previous work has proposed methods to ensure the usability of overlays even
while under attack.  A possible approach is to replicate public information
within a user's profile overlay thus providing an alternative directory overlay
for querying prior to using the public directory overlay.

\section{Figures and Tables}

\begin{figure}[ht]
\centering
\epsfig{file=figs/subrings.eps, width=3.12in}
\caption[An example social overlay network]{An example social overlay network.
Alice has a friendship with Bob and Carol, hence both are members of her
profile overlay. Bob has a friendship with Alice and Dave but not Carol; hence
Alice and Dave are members of his profile overlay, while Carol is not.  Each
peer has many overlay memberships but a single root represented by dashed lines
in various shades of gray.  For clarity, overlay shortcut connections are not
shown.}
\label{fig:system}
\end{figure}

