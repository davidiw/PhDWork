\chapter{CONCLUSIONS}
\label{chap:conclusion}
This work brings significant advances to the usability of VPNs by describing,
implementing, and evaluating methods that provide simplified configuration of
both the local and network components of a VPN.  The use of common network
protocols enables OS independent VPN system supporting many different local VPN
configurations.  Structured overlays enable scalabe design of VPNs but have
inherent security issue; virtual private overlays enables users in network
constrained environments to use a public overlay to create a secure overlay.
Web interfaces make the user configuration of these systems reasonable for even
non-experts.  Decentralized relays address concerns when network constraints
limit direct communication between peers.  The future work addresses
performance issues inherent to the use of direct IP communication over P2P and
enabling the use of VPNs in unprivileged locations.

Completed components of this work include:
\begin{itemize}
\item \textbf{Virtual Private Overlays} - secure, self-configuring overlay as
the basis for a structured overlay VPN.
\item \textbf{Group environments} - User-friendly environments to generate
files to ease configuration of complex systems.
\item \textbf{Local VPN configuration} - VPN architect supporting Interface,
Router, and a novel Hybrid mode for various environments.
\item \textbf{Relays} - to enable two-hop connections between peers that cannot
form direct connections.
\end{itemize}

The VPN approach described herein has been used to construct real systems,
such as the GroupVPN~\cite{gridappliance} and a SocialVPN~\cite{cops08}.  The
GroupVPN has been used to construct a Grid Appliance~\cite{grid_appliance} that
enables the creation of distributed, decentralized, dynamic computing grids.
Over the past 2 years, I have been the student lead in an active grid deployed
for computer architecture research, Archer~\cite{archer}.  Archer currently
spans four universities with 500 resources.  We have had 100s of users connect
seamlessly to these resources from many locations including home, school, and
hotels.  A PlanetLab backend distributed across over 600 resources provides
near constant overlay uptime for Archer and external users.  External users
include classes and groups at other universities.  Most recently, a grid at La
Jolla Institute for Allergy and Immunology went live with minimal communication
with our group.  Researchers at the Clemson University and Purdue have opted
for this approach over centralized VPNs as the basis of their future
distributed compute clusters and have actively tested networks of over 1000
nodes.
