\chapter{CONCLUSIONS}
\label{chap:conclusion}

This work brings significant advances to the usability of VPNs through
significant background study, understanding important practical applications,
and verification in both simulation and real deployments.  The architecture
explored herein provides a general framework for creating VPNs that has
contributed to various end-point and overlay configurations useful for both
large and small scale deployments for group or personal use.

In order to support a completely ad-hoc, decentralized VPN, users begin by
connecting to a public overlay, such as XMPP or Kademlia, in order to discover
other users.  After exchanging their information via these mediums, peers can
establish direct communication links with each other and with other peers
already in the VPN.  Peers can exchange or obtain trusted identities using
established peer or group relationships in existing social networks.

Because most peers are behind NATs and firewalls, this dissertation covers
methods, which allow peers to use existing overlays to bootstrap through NATs
and firewalls.  This can mean using a third-party system until a peer on a
public IP address comes online, or more likely, using a service to obtain a
public IP and port mapping for the peers private address.  When peers cannot
directly establish direct links, the overlay still provides the ability to
route messages between peers.  Routing messages across the overlay can incur
significant overhead and is not optimized for any specific purpose.  To remedy
this, I have established a mechanism to create two-hop links between peers
emphasized on the latency between the peers.

This work describes two novel mechanisms for handling address assignments
inside a VPN:  using a DHT with atomic capabilities as well as independent
networks with explicit links based upon social connections.  The DHT approach
allows for highly scalable systems in comparison to other approaches that
require state to be manually spread across the system or through the use of
broadcast mechanisms.  By making the network addresses dependent on social
links and independent of the actual overlay, peers need not worry about address
collisions.  Furthermore, this work explores means to transparently migrate
resources using DHT style addressing even when those resources are connected
via the VPN router.

Existing approaches to VPN placement use either interface or router models.
Interface models can easily be constructed to be transparent, whereas existing
router models have no such features.  Through the use of network protocols
support in network stacks found in common operating systems, mechanisms for
support transparent configuration of both interface and routing models have
been detailed.  Where routers are desirable due to performance, and interface
is attractive for security purposes, hybrid provides a middle ground that
combines these two aspects to support high-performance, though secure virtual
networking.

Because attempting to verify the system in every possible environment after
making both minor and major modifications, I have employed a built-in
self-simulating environment into Brunet.  The application has reduced the time
necessary to develop, evaluate, and debug new contributions as well as
reproduce bugs and protect from having them reoccur.  In the context of this
dissertation, it has been used to verify and motivate the necessity for on
demand as opposed to passive connections, as well as the relay, bootstrapping,
and security work.

This work has been the corner stone in a real system used to provide ad-hoc
grid computing named the Grid Appliance~\cite{gridappliance}, which has been
realized in a voluntary computing grid for computer architecture research
called Archer~\cite{archer}.  Archer currently spans six universities with over
600 resources.  Over 100s of users have connected seamlessly to these resources
from many locations including home, school, and hotels.  A PlanetLab back end
distributed across over 600 resources provides near constant overlay uptime for
Archer and external users.  External users include classes and groups at other
universities.  Most recently, a grid at La Jolla Institute for Allergy and
Immunology went live with minimal communication with our group.  Researchers at
the Clemson University and Purdue have opted for this approach over centralized
VPNs as the basis of their future distributed compute clusters and have
actively tested networks of over 1000 nodes.

The majority of this dissertation focused on services to enable user-friendly
VPNs, during the design and evaluation, it became apparent that there still
exist significant deficiencies in the design described herein.  The following
are important research topics that are left as future research topics.  During
the evaluation of packet drop rates across the Internet and in particularly
PlanetLab, it is apparent that recursive packet routing will not scale well
especially when dealing with non-negligible traffic.  Another aspect of limited
scalability is presented when using a single UDP socket multiplexed across
potentially many VPN connections.  Each UDP socket has a small buffer
associated with it, when that buffer is exhausted sends either either become
blocking or are thrown away, depending on usage.  In terms of VPNs, there still
exists a wide gap between IPv6 and IPv4, my work could be used to assist in
deploying IPv6 to IPv4 tunnels, which unlike existing approaches, have natural
fail over support and efficient paths between source and destination.  Finally,
the last major draw back to the Grid Appliance is the necessity for a
centralized scheduler / management component.  Ideally, this could be handled
in decentralized means with trust, a user rank system for priority, ability to
handle simultaneous scheduling of tasks, and limiting the reliance of a node
being online to receive results for tasks.
