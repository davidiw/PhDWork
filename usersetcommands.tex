% user defined commands %
% Here is where you define optional commands such as macros, new commands,
% and new environments to be used in your paper

% optional command to prevent a word from breaking across a line %
\hyphenchar\font=-1

% UF template specific commands %
% Commands to produce proper bullet list (creates the \uflistb and \bitem commands) %
\newenvironment{uflistb}[1]{\begin{hangparas}{.34in}{1}}{\end{hangparas}} %
\newcommand{\bitem}{\noindent\singlespacing\labelitemi\hspace{.25in}} %

% Commands for enumerated lists (creates the \uflistn and \nitem commands) %
\newcounter{ufcount}%
\newenvironment{uflistn}[1]{\begin{hangparas}{.36in}{1}\setcounter{ufcount}{1}}{\end{hangparas}} %
\renewcommand{\labelitemii}{\arabic{ufcount}.} %
\newcommand{\nitem}{\noindent\singlespacing\labelitemii\hspace{.23in}\addtocounter{ufcount}{1}} %
\newcommand{\labelbitemi}{\labelitemi}
\newcommand{\labelbitemii}{\labelitemii}
\newcommand{\labelbitemiii}{\labelitemiii}
\newcommand{\labelbitemiv}{\labelitemiv}
% Shorcut commands for misc stuff %

% Commands to produce proper bullet list
\newlength{\widthOfItem}
\let\Itemize=\itemize
\let\endItemize=\enditemize
\renewenvironment{itemize}{%
	\begin{Itemize}
		\setlength{\itemsep}{0.5\baselineskip}
		\setlength{\labelwidth}{2em}
		\setlength{\listparindent}{.32in}%
		\setlength{\leftmargin}{.32in}
		\setlength{\rightmargin}{0in}
		\settowidth{\widthOfItem}{\labelitemi}
		\setlength{\labelsep}{\leftmargin-\widthOfItem}
		\renewcommand{\labelitemii}{--}
		\singlespacing}{%
	\end{Itemize}}

% shortcut for setting up inserting \prime command in mathmode to avoid errors %
\newcommand{\p}{^{\prime}}

% shortcuts for prime color text
\newcommand{\red}{\textcolor[rgb]{1.00,0.00,0.00}}
\newcommand{\green}{\textcolor[rgb]{0.00,1.00,0.00}}
\newcommand{\blue}{\textcolor[rgb]{0.00,0.00,1.00}}

% Shorcut commands for mathmatical formulas %

\newcommand{\latex}{\LaTeX 2\ensuremath{\epsilon}}

% THEOREM Environments ---------------------------------------------------
%These environments are provided as a convenience - feel free to modify if needed

\newtheorem{theorem}{Theorem}[chapter]%To link the theorem to each chapter uncomment the chapter option
\newtheorem{lemma}{Lemma}%[theorem]% To link each lemma to a theorem uncomment the theorem option
\newtheorem{corollary}{Corollary}%[theorem]% To link each corollary to a theorem uncomment the theorem option
% to link a corollary to a chapter change the theorem option to chapter
\newtheorem{definition}{Definition}%[chapter] %the same is true for both definitions and assumptions
\newtheorem{assumption}{Assumption}%[chapter] %
\newtheorem{proposition}{Proposition}[chapter]


%These were some user commands I've run across that I thought some might want to incorporate into their work
%\newcommand{\bdm}{
 %   \begin{displaymath}}

%\newcommand{\edm}{
%    \end{displaymath}}

%\newcommand{\be}{
%    \begin{equation}}

%\newcommand{\ee}{
%    \end{equation}}

%\newcommand{\bea}{
 %   \begin{eqnarray}}

%\newcommand{\eea}{
%    \end{eqnarray}}
